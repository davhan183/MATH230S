\documentclass[12pt]{article}

\usepackage{color}
%\input{rgb}
%----------Packages----------
\usepackage{amsmath}
\usepackage{amssymb}
\usepackage{amsthm}
\usepackage{amsrefs}
\usepackage{dsfont}
\usepackage{enumerate}
\usepackage{hyperref}
\usepackage{mathrsfs}
\usepackage{stmaryrd}
\usepackage{tikz}
	\usetikzlibrary{matrix}
\usepackage[all]{xy}
\usepackage[mathcal]{eucal}
\usepackage{verbatim}  %%includes comment environment
\usepackage{fullpage}  %%smaller margins
%----------Commands----------

%%penalizes orphans
\clubpenalty=9999
\widowpenalty=9999


%% blackboard bold math capitals
\newcommand{\bbA}{\mathbb{A}}
\newcommand{\bbB}{\mathbb{B}}
\newcommand{\bbC}{\mathbb{C}}
\newcommand{\bbD}{\mathbb{D}}
\newcommand{\bbE}{\mathbb{E}}
\newcommand{\bbF}{\mathbb{F}}
\newcommand{\bbG}{\mathbb{G}}
\newcommand{\bbH}{\mathbb{H}}
\newcommand{\bbI}{\mathbb{I}}
\newcommand{\bbJ}{\mathbb{J}}
\newcommand{\bbK}{\mathbb{K}}
\newcommand{\bbL}{\mathbb{L}}
\newcommand{\bbM}{\mathbb{M}}
\newcommand{\bbN}{\mathbb{N}}
\newcommand{\bbO}{\mathbb{O}}
\newcommand{\bbP}{\mathbb{P}}
\newcommand{\bbQ}{\mathbb{Q}}
\newcommand{\bbR}{\mathbb{R}}
\newcommand{\bbS}{\mathbb{S}}
\newcommand{\bbT}{\mathbb{T}}
\newcommand{\bbU}{\mathbb{U}}
\newcommand{\bbV}{\mathbb{V}}
\newcommand{\bbW}{\mathbb{W}}
\newcommand{\bbX}{\mathbb{X}}
\newcommand{\bbY}{\mathbb{Y}}
\newcommand{\bbZ}{\mathbb{Z}}


\renewcommand{\phi}{\varphi}

\renewcommand{\emptyset}{\O}

\providecommand{\abs}[1]{\lvert #1 \rvert}
\providecommand{\norm}[1]{\lVert #1 \rVert}


\providecommand{\ar}{\rightarrow}
\providecommand{\arr}{\longrightarrow}

\renewcommand{\_}[1]{\underline{ #1 }}


\DeclareMathOperator{\ext}{ext}



%----------Theorems----------

\newtheorem{theorem}{Theorem}[section]
\newtheorem{proposition}[theorem]{Proposition}
\newtheorem{lemma}[theorem]{Lemma}
\newtheorem{corollary}[theorem]{Corollary}


\newtheorem{axiom}{Axiom}


\theoremstyle{definition}
\newtheorem{definition}[theorem]{Definition}
\newtheorem{exercise}[theorem]{Exercise}
\newtheorem{example}[theorem]{Example}
\newtheorem{remark}[theorem]{Remark}
\newtheorem{notation}[theorem]{Notation}
\newtheorem{warning}[theorem]{Warning}


\numberwithin{equation}{subsection}


%----------Title-------------


\begin{document}


\begin{center}
{\large MATH/STAT 230S, SCRIPT 4: INDEPENDENT EVENTS} \\ 
\vspace{.2in}  

\end{center}

% % % % % The counter sets us as having completed section 3. So we'll start this script on section 4 (since it's the fourth script)

\setcounter{section}{3}



\section{Independent Events}
Conditional probabilities define independent events. Heuristically, events are independent if knowledge of one event does not impact the probability of the other.
\begin{definition}
Two events $A$ and $B$ are \emph{independent} if $\bbP(A)=\bbP(A|B)$, or $\bbP(B)=\bbP(B|A)$, or $\bbP(A\cap B)=\bbP(A)\bbP(B)$
\end{definition}

\begin{example}
The events of getting heads on a coin toss and a 6 in a fair 6-sided die are independent. However, the events of drawing an Ace from a standard deck of cards and then the drawing another Ace from the same deck, without replacements, are dependent.
\end{example}

\begin{definition}
Three events, $A$, $B$, and $C$ are \emph{independent} or \emph{mutually independent}. If both of following hold
  \begin{enumerate}
    \item $A$ and $B$ are independent: \[\bbP(B|A)=\bbP(B|A^c)=\bbP(B)\]
     
     \item $C$ does not depend on the occurrence of $A$ or $B$: \[\bbP(C)=\bbP(C|A\cap B)=\bbP(C|A^c\cap B)=\bbP(C|A\cap B^c)\]
     \end{enumerate}
     Alternately, you can check that three events are independent by checking both of the following
     \begin{enumerate}
     \item All three events are pairwise independent (any two you pick are independent).
     \item $\mathbb{P}(A\cap B\cap C)=\mathbb{P}(A)\mathbb{P}(B)\mathbb{P}(C)$
  \end{enumerate}
\end{definition}
    If three events are independent, then the following multiplication rule applies:
  
    \begin{theorem}
      (Multiplication Rule for Three Independent Events)
      \[\bbP(A\cap B\cap C)=\bbP(A)\bbP(B)\bbP(C)\]
    \end{theorem}

  \begin{example}
    Given we are rolling a fair 6-sided twice, let $A$ be the event that you roll a 1 on the first die, $B$ be the event that you roll a 2 on the first die, and $C$ be the event that you roll a sum of 0 where
    \begin{align*}
      \bbP(A)&=\frac{1}{6} \\
      \bbP(B)&=\frac{1}{6} \\
      \bbP(C)&=\frac{0}{36}=0 \\
      \bbP(A\cap B\cap C)&=\bbP(A)\bbP(B)\bbP(C) \\
        &=\frac{1}{6}\cdot \frac{1}{6}\cdot 0 = 0
    \end{align*}
    This demonstrates that the multiplication rule does not imply independence as $\bbP(A\cap B\cap C)$ is 0 as rolling a sum of 0 is . \\
  \end{example}

	\begin{example}
	Let $\Omega$ be an outcome space associated to rolling one die twice. Let $A$ be the event that the first roll is even. Let $B$ be the event that the second roll is even. Let $C$ be the event that the sum of the rolls is $7$.\\
	Are these three events (mutually) independent? Are they \textit{pairwise} independent? (pairwise independence means $A$ and $B$ are independent, $A$ and $C$ and independent, and $B$ and $C$ are independent)
		\begin{proof}[Solution]
		%insert your solution
    Given
    \begin{center}
      $\Omega=\Bigg\{$
      \begin{tabular}{ c c c c c c } 
        11, & 12, & 13, & 14, & 15, & 16, \\
        21, & 22, & 23, & 24, & 25, & 26, \\
        31, & 32, & 33, & 34, & 35, & 36, \\
        41, & 42, & 43, & 44, & 45, & 46, \\
        51, & 52, & 53, & 54, & 55, & 56, \\
        61, & 62, & 63, & 64, & 65, & 66
      \end{tabular}
      $\Bigg\}$ \\
    \end{center}
    \begin{align*}
      \bbP(A)&=\frac{18}{36}=\frac{1}{2} \\
      \bbP(B)&=\frac{18}{36}=\frac{1}{2} \\
      \bbP(C)&=\frac{6}{36}=\frac{1}{6} \\
      \bbP(A\cap B\cap C)&=\bbP(A)\bbP(B)\bbP(C) \\
        &=\frac{1}{2}\cdot \frac{1}{2}\cdot \frac{1}{6}=\frac{1}{24} \neq \frac{0}{36}
    \end{align*}
    As such, events A, B, and C are not mutually independent. \\
    \begin{align*}
      \bbP(A|B)&=\frac{9}{18}=\frac{1}{2}=\bbP(A) \\
      \bbP(C|A)&=\frac{3}{18}=\frac{1}{6}=\bbP(C) \\
      \bbP(C|B)&=\frac{3}{18}=\frac{1}{6}=\bbP(C)
    \end{align*}
    Thus, A, B, and C are all pairwise independent.
		\end{proof}
	\end{example}
	
\end{document}